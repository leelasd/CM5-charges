%%%%%%%%%%%%%%%%%%%%%%%%%%%%%%%%%%%%%%%%%
%  My documentation report
%  Objetive: Explain what I did and how, so someone can continue with the investigation
%
% Important note:
% Chapter heading images should have a 2:1 width:height ratio,
% e.g. 920px width and 460px height.
%
%%%%%%%%%%%%%%%%%%%%%%%%%%%%%%%%%%%%%%%%%

%----------------------------------------------------------------------------------------
%	PACKAGES AND OTHER DOCUMENT CONFIGURATIONS
%----------------------------------------------------------------------------------------

\documentclass[11pt,fleqn]{book} % Default font size and left-justified equations

\usepackage[top=3cm,bottom=3cm,left=3.2cm,right=3.2cm,headsep=10pt,letterpaper]{geometry} % Page margins

\usepackage{xcolor,lipsum} % Required for specifying colors by name
\definecolor{ocre}{RGB}{51,102,0} 
\definecolor{lightgray}{RGB}{229,229,229} 
% Font Settings
\usepackage{avant} % Use the Avantgarde font for headings
%\usepackage{times} % Use the Times font for headings
\usepackage{mathptmx} % Use the Adobe Times Roman as the default text font together with math symbols from the Sym­bol, Chancery and Com­puter Modern fonts
\usepackage{listings}

\usepackage{microtype} % Slightly tweak font spacing for aesthetics
\usepackage[utf8]{inputenc} % Required for including letters with accents
\usepackage[T1]{fontenc} % Use 8-bit encoding that has 256 glyphs

%%
%%%%%%%%%%%%%%%% Python stuff

\definecolor{codegreen}{rgb}{0,0.6,0}
\definecolor{codegray}{rgb}{0.5,0.5,0.5}
\definecolor{codepurple}{rgb}{0.58,0,0.82}
\definecolor{backcolour}{rgb}{0.95,0.95,0.92}

\lstdefinestyle{mystyle}{
    backgroundcolor=\color{backcolour},
    commentstyle=\color{codegreen},
    keywordstyle=\color{magenta},
    numberstyle=\small\color{codegray},
    stringstyle=\color{codepurple},
    basicstyle=\small,
    breakatwhitespace=false,
    breaklines=true,
    captionpos=b,
    keepspaces=true,
    numbers=left,
    numbersep=5pt,
    showspaces=false,
    showstringspaces=false,
    showtabs=false,
    tabsize=2
}

\lstset{style=mystyle}
%%
% MATHS PACKAGE
\usepackage{amsmath,tikz}
\usetikzlibrary{matrix}
\newcommand*{\horzbar}{\rule[0.05ex]{2.5ex}{0.5pt}}
\usepackage{calc}

% VERBATIM PACKAGE
\usepackage{verbatim}

% Bibliography
\usepackage[style=alphabetic,sorting=nyt,sortcites=true,autopunct=true,babel=hyphen,hyperref=true,abbreviate=false,backref=true,backend=biber]{biblatex}
\addbibresource{bibliography.bib} % BibTeX bibliography file
\defbibheading{bibempty}{}

\input{structure} % Insert the commands.tex file which contains the majority of the structure behind the template

\begin{document}

\let\cleardoublepage\clearpage

%----------------------------------------------------------------------------------------
%	TITLE PAGE
%----------------------------------------------------------------------------------------

\begingroup
\thispagestyle{empty}
\AddToShipoutPicture*{\put(0,0){\includegraphics[scale=1.25]{v}}} % Image background
\centering
\vspace*{5cm}
\par\normalfont\fontsize{35}{35}\sffamily\selectfont
\textbf{MODELE LINEAIRE A EFFETS MIXTES\\ THEORIE \& APPLICATION }\\
{\LARGE }\par % Book title
\vspace*{1cm}
{\Huge Amin EL GAREH et Bezeid CHEICK-MOHAMED-LMAMI}\par % Author name
\endgroup

%----------------------------------------------------------------------------------------
%	COPYRIGHT PAGE
%----------------------------------------------------------------------------------------

\newpage
~\vfill
\thispagestyle{empty}

%\noindent Copyright \copyright\ 2014 Andrea Hidalgo\\ % Copyright notice

\noindent \textsc{Projet Janvier-Mars 2015, Université de Bourgogne}\\

\noindent Ce projet a été encadré par Hervé CARDOT.\\ % License information

\noindent \textit{Publié le 30 Mars 2015} % Printing/edition date

%----------------------------------------------------------------------------------------
%	TABLE OF CONTENTS
%----------------------------------------------------------------------------------------

\chapterimage{pano-5.jpg} % heading image

\pagestyle{empty} % No headers

\renewcommand\contentsname{Table des Matières}
\renewcommand{\bibname}{Bibliographie}
\tableofcontents% Print the table of contents itself

%\cleardoublepage % Forces the first chapter to start on an odd page so it's on the right

\pagestyle{fancy} % Print headers again

%----------------------------------------------------------------------------------------
%	CHAPTER 1
%----------------------------------------------------------------------------------------

\chapterimage{pano-5.jpg} % Chapter heading image

% ----------------------------------------------------------------------------------------
% 	BIBLIOGRAPHY
% ----------------------------------------------------------------------------------------
%----------------------------------------------------------------------------------------
%	INTRODUCTION SECTION
%----------------------------------------------------------------------------------------

\chapter*{Introduction} % Introduction chapter suppressed from the table of contents

\begin{quote}
This is one of my finer quotations.\\
--John Smith
\end{quote}

This is a great place to write an introduction or prologue\footnote{You can even use a footnote to seem smarter}.

%----------------------------------------------------------------------------------------
%	BOOK PART
%----------------------------------------------------------------------------------------

\part{Python recipies}

%----------------------------------------------------------------------------------------
%	CHAPTER ONE
%----------------------------------------------------------------------------------------

\chapter{Linear Regression}
Packages required to run this code
\begin{description}
  \item[pandas] for reading csv files\footnote{data not shown as tables} format
  \item[scipy] for doing linear regression analysis and obtaining the statistics
  \item[matplotlib] for making the plots
\end{description}
\pagebreak
\begin{figure}
\includegraphics[scale=0.55]{../python_recipies/GBSA_comp.pdf}
\caption{Linear regression analysis has been performed for two sets of data and the resulting model is shown in the legends of each figure}
\end{figure}
\lstinputlisting[language=Python]{../python_recipies/Lin_reg_ghyd.py}



%----------------------------------------------------------------------------------------
%	CHAPTER TWO
%----------------------------------------------------------------------------------------

\chapter{Heat Maps}



%----------------------------------------------------------------------------------------
%	CHAPTER THREE
%----------------------------------------------------------------------------------------

\chapter{Barplots}
Packages required to run this code
\begin{description}
  \item[pandas] for reading ``Hvap.csv\footnote{contains both the raw and devation data required for plot}" format
  \item[numpy] for creating and manipulating vectors
  \item[matplotlib] for making the plots
\end{description}
\begin{table}
\caption{Data to be plotted using bar plots}
\begin{tabular}{lcccc}
\toprule
             Molecules &  OPLS &  CM1A &  CM5 &  Expt \\
\midrule
          Acetic acid &      12.26 &      13.52 &     14.46 &      12.49 \\
              Acetone &       7.23 &       7.74 &      8.92 &       7.48 \\
         Acetonitrile &       7.57 &       7.63 &      9.76 &       8.01 \\
              Aniline &      11.88 &      16.41 &     14.61 &      12.60 \\
         Benzonitrile &      12.52 &      14.45 &     15.49 &      12.54 \\
          Cyclohexane &       7.56 &       7.64 &      7.61 &       7.86 \\
         Diethylamine &       7.68 &       7.54 &      7.46 &       7.48 \\
        Diethyl ether &       6.90 &       7.01 &      7.22 &       6.56 \\
N,N-dimethylacetamide &      13.44 &      14.34 &     15.57 &      11.75 \\
          Ethanethiol &       6.67 &       6.48 &      6.68 &       6.58 \\
              Ethanol &      10.29 &       9.06 &     10.19 &      10.11 \\
                Furan &       6.91 &       8.01 &      7.17 &       6.56 \\
               Hexane &       7.54 &       7.48 &      7.34 &       7.54 \\
             Methanol &       9.00 &       7.60 &      8.84 &       8.95 \\
       Methyl acetate &       7.99 &      10.00 &     10.12 &       7.72 \\
          Nitroethane &       9.78 &      14.16 &     11.72 &       9.94 \\
    N-methylacetamide &      13.87 &      16.12 &     19.06 &      13.30 \\
               Phenol &      14.58 &      14.63 &     14.30 &      13.82 \\
          Propylamine &       7.90 &       8.93 &      7.23 &       7.47 \\
             Pyridine &       9.76 &      11.16 &     11.16 &       9.61 \\
              Pyrrole &      10.32 &      13.81 &     12.37 &      10.80 \\
      Tetrahydrofuran &       7.52 &       7.66 &      8.08 &       7.61 \\
\bottomrule
\end{tabular}
\end{table}
\pagebreak 
\lstinputlisting[language=Python,caption={Bar plot of the data shown in Table above}]{../python_recipies/bar_plot_hvap.py}
\begin{figure}
\includegraphics[scale=0.80]{../python_recipies/Tesh_hvap.pdf}
\caption{Data in table above is plotted where instead of raw data, deviations from experiments for each method is plotted}
\end{figure}
\pagebreak 
\lstinputlisting[language=Python,caption={Barplot liquid properties using CM5 charges with different scale factors}]{../python_recipies/BARS_all.py}
\begin{figure}
\includegraphics[scale=0.80]{../python_recipies/Thh.pdf}
\caption{Data in table above is plotted where instead of raw data, deviations from experiments for each method is plotted}
\end{figure}
\pagebreak 
\lstinputlisting[language=Python,caption={Multiple bar plots in matplotlib}]{../python_recipies/Multi_Bar.py}
\begin{figure}
\includegraphics[scale=0.80]{../python_recipies/Multi_Bar.pdf}
\caption{Data in table above is plotted where instead of raw data, deviations from experiments for each method is plotted}
\end{figure}
\part{R recipies}

\end{document}
