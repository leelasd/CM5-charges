%%%%%%%%%%%%%%%%%%%%%%%%%%%%%%%%%%%%%%%%%
% eBook 
% LaTeX Template
% Version 1.0 (29/12/14)
%
% This template has been downloaded from:
% http://www.LaTeXTemplates.com
%
% Original author:
% Luis Cobo (luiscobogutierrez@gmail.com) with extensive modifications by:
% Vel (vel@latextemplates.com)
%
% License:
% CC BY-NC-SA 3.0 (http://creativecommons.org/licenses/by-nc-sa/3.0/)
%
%%%%%%%%%%%%%%%%%%%%%%%%%%%%%%%%%%%%%%%%%

%----------------------------------------------------------------------------------------
%	DOCUMENT CONFIGURATIONS AND INFORMATION
%----------------------------------------------------------------------------------------

\documentclass[oneside,11pt]{memoir} % Font size

\input{structure.tex} % Include the file that specifies the document structure and layout

\title{Data Visualization in R and python} % Book title
\author{Leela S. Dodda} % Author
\newcommand{\edition}{First Edition} % Book edition

%----------------------------------------------------------------------------------------

\begin{document}

%----------------------------------------------------------------------------------------
%	TITLE PAGE
%----------------------------------------------------------------------------------------

\thispagestyle{empty} % Suppress page numbering
\ThisCenterWallPaper{1.12}{littlered.jpg} % Add the background image, the first argument is the scaling - adjust this as necessary so the image fits the entire page

\begin{tikzpicture}[remember picture,overlay]
\node [rectangle, rounded corners, fill=white, opacity=0.75, anchor=south west, minimum width=5cm, minimum height=6cm] (box) at (-0.5,-10) (box){}; % White rectangle - "minimum width/height" adjust the width and height of the box; "(-0.5,-10)" adjusts the position on the page
\node[anchor=west, color01, xshift=-2.2cm, yshift=-0.4cm, text width=4.2cm, font=\sffamily\scriptsize] at (box.north){\edition}; % "Text width" adjusts the wrapping width, "xshift/yshift" adjust the position relative to the white rectangle
\node[anchor=west, color01, xshift=-2.2cm, yshift=-2cm, text width=4.2cm, font=\sffamily\bfseries\scshape\Large] at (box.north){\thetitle}; % "Text width" adjusts the wrapping width, "xshift/yshift" adjust the position relative to the white rectangle
\node[anchor=west, color01, xshift=-2.2cm, yshift=-5cm, text width=4.2cm, font=\sffamily\bfseries] at (box.north){\theauthor}; % "Text width" adjusts the wrapping width, "xshift/yshift" adjust the position relative to the white rectangle
\end{tikzpicture}

\newpage % Make sure the following content is on a new page

%----------------------------------------------------------------------------------------
%	TABLE OF CONTENTS
%----------------------------------------------------------------------------------------

\tableofcontents % Prints the table of contents

%----------------------------------------------------------------------------------------
%	INTRODUCTION SECTION
%----------------------------------------------------------------------------------------

\chapter*{Introduction} % Introduction chapter suppressed from the table of contents

\begin{quote}
This is one of my finer quotations.\\
--John Smith
\end{quote}

This is a great place to write an introduction or prologue\footnote{You can even use a footnote to seem smarter}.

%----------------------------------------------------------------------------------------
%	BOOK PART
%----------------------------------------------------------------------------------------

\part{Python recipies}

%----------------------------------------------------------------------------------------
%	CHAPTER ONE
%----------------------------------------------------------------------------------------

\chapter{Linear Regression}
Packages required to run this code
\begin{description}
  \item[pandas] for reading csv files\footnote{data not shown as tables} format
  \item[scipy] for doing linear regression analysis and obtaining the statistics
  \item[matplotlib] for making the plots
\end{description}

\lstinputlisting[language=Python]{python_recipies/Lin_reg_ghyd.py}
\begin{figure}
\includegraphics[scale=0.3]{python_recipies/GBSA_comp.pdf}
\caption{Linear regression analysis has been performed for two sets of data and the resulting model is shown in the legends of each figure}
\end{figure}



%----------------------------------------------------------------------------------------
%	CHAPTER TWO
%----------------------------------------------------------------------------------------

\chapter{Heat Maps}



%----------------------------------------------------------------------------------------
%	CHAPTER THREE
%----------------------------------------------------------------------------------------

\chapter{Barplots}
Packages required to run this code
\begin{description}
  \item[pandas] for reading ``Hvap.csv\footnote{contains both the raw and devation data required for plot}" format
  \item[numpy] for creating and manipulating vectors
  \item[matplotlib] for making the plots
\end{description}
\begin{table}
\caption{Data to be plotted using bar plots}
\begin{tiny}
\begin{tabular}{lcccc}
\toprule
             Molecules &  OPLS &  CM1A &  CM5 &  Expt \\
\midrule
          Acetic acid &      12.26 &      13.52 &     14.46 &      12.49 \\
              Acetone &       7.23 &       7.74 &      8.92 &       7.48 \\
         Acetonitrile &       7.57 &       7.63 &      9.76 &       8.01 \\
              Aniline &      11.88 &      16.41 &     14.61 &      12.60 \\
         Benzonitrile &      12.52 &      14.45 &     15.49 &      12.54 \\
          Cyclohexane &       7.56 &       7.64 &      7.61 &       7.86 \\
         Diethylamine &       7.68 &       7.54 &      7.46 &       7.48 \\
        Diethyl ether &       6.90 &       7.01 &      7.22 &       6.56 \\
N,N-dimethylacetamide &      13.44 &      14.34 &     15.57 &      11.75 \\
          Ethanethiol &       6.67 &       6.48 &      6.68 &       6.58 \\
              Ethanol &      10.29 &       9.06 &     10.19 &      10.11 \\
                Furan &       6.91 &       8.01 &      7.17 &       6.56 \\
               Hexane &       7.54 &       7.48 &      7.34 &       7.54 \\
             Methanol &       9.00 &       7.60 &      8.84 &       8.95 \\
       Methyl acetate &       7.99 &      10.00 &     10.12 &       7.72 \\
          Nitroethane &       9.78 &      14.16 &     11.72 &       9.94 \\
    N-methylacetamide &      13.87 &      16.12 &     19.06 &      13.30 \\
               Phenol &      14.58 &      14.63 &     14.30 &      13.82 \\
          Propylamine &       7.90 &       8.93 &      7.23 &       7.47 \\
             Pyridine &       9.76 &      11.16 &     11.16 &       9.61 \\
              Pyrrole &      10.32 &      13.81 &     12.37 &      10.80 \\
      Tetrahydrofuran &       7.52 &       7.66 &      8.08 &       7.61 \\
\bottomrule
\end{tabular}
\end{tiny}
\end{table}
\pagebreak 
\lstinputlisting[language=Python,caption={Bar plot of the data shown in Table above}]{python_recipies/bar_plot_hvap.py}
\begin{figure}
\includegraphics[scale=0.40]{python_recipies/Tesh_hvap.pdf}
\caption{Data in table above is plotted where instead of raw data, deviations from experiments for each method is plotted}
\end{figure}
\pagebreak 
\lstinputlisting[language=Python,caption={Barplot liquid properties using CM5 charges with different scale factors}]{python_recipies/BARS_all.py}
\begin{figure}
\includegraphics[scale=0.30]{python_recipies/Thh.pdf}
\caption{Data in table above is plotted where instead of raw data, deviations from experiments for each method is plotted}
\end{figure}
\pagebreak 
\lstinputlisting[language=Python,caption={Multiple bar plots in matplotlib}]{python_recipies/Multi_Bar.py}
\begin{figure}
\includegraphics[scale=0.40]{python_recipies/Multi_Bar.pdf}
\caption{Data in table above is plotted where instead of raw data, deviations from experiments for each method is plotted}
\end{figure}
\pagebreak 
\lstinputlisting[language=Python,caption={Multiple bar plots in matplotlib}]{python_recipies/Errors_Multi_Bar.py}
\begin{figure}
\includegraphics[scale=0.40]{python_recipies/Ers_Multi_Bar.pdf}
\caption{Data in table above is plotted where instead of raw data, deviations from experiments for each method is plotted}
\end{figure}
\part{R recipies}

\end{document}
